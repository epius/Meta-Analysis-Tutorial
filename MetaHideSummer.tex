\documentclass[]{article}
\usepackage{lmodern}
\usepackage{amssymb,amsmath}
\usepackage{ifxetex,ifluatex}
\usepackage{fixltx2e} % provides \textsubscript
\ifnum 0\ifxetex 1\fi\ifluatex 1\fi=0 % if pdftex
  \usepackage[T1]{fontenc}
  \usepackage[utf8]{inputenc}
\else % if luatex or xelatex
  \ifxetex
    \usepackage{mathspec}
  \else
    \usepackage{fontspec}
  \fi
  \defaultfontfeatures{Ligatures=TeX,Scale=MatchLowercase}
\fi
% use upquote if available, for straight quotes in verbatim environments
\IfFileExists{upquote.sty}{\usepackage{upquote}}{}
% use microtype if available
\IfFileExists{microtype.sty}{%
\usepackage{microtype}
\UseMicrotypeSet[protrusion]{basicmath} % disable protrusion for tt fonts
}{}
\usepackage[margin=1in]{geometry}
\usepackage{hyperref}
\hypersetup{unicode=true,
            pdftitle={Tutorial: Meta-Analysis of Proportion in R},
            pdfauthor={Pius S. Ekong},
            pdfborder={0 0 0},
            breaklinks=true}
\urlstyle{same}  % don't use monospace font for urls
\usepackage{color}
\usepackage{fancyvrb}
\newcommand{\VerbBar}{|}
\newcommand{\VERB}{\Verb[commandchars=\\\{\}]}
\DefineVerbatimEnvironment{Highlighting}{Verbatim}{commandchars=\\\{\}}
% Add ',fontsize=\small' for more characters per line
\usepackage{framed}
\definecolor{shadecolor}{RGB}{248,248,248}
\newenvironment{Shaded}{\begin{snugshade}}{\end{snugshade}}
\newcommand{\KeywordTok}[1]{\textcolor[rgb]{0.13,0.29,0.53}{\textbf{#1}}}
\newcommand{\DataTypeTok}[1]{\textcolor[rgb]{0.13,0.29,0.53}{#1}}
\newcommand{\DecValTok}[1]{\textcolor[rgb]{0.00,0.00,0.81}{#1}}
\newcommand{\BaseNTok}[1]{\textcolor[rgb]{0.00,0.00,0.81}{#1}}
\newcommand{\FloatTok}[1]{\textcolor[rgb]{0.00,0.00,0.81}{#1}}
\newcommand{\ConstantTok}[1]{\textcolor[rgb]{0.00,0.00,0.00}{#1}}
\newcommand{\CharTok}[1]{\textcolor[rgb]{0.31,0.60,0.02}{#1}}
\newcommand{\SpecialCharTok}[1]{\textcolor[rgb]{0.00,0.00,0.00}{#1}}
\newcommand{\StringTok}[1]{\textcolor[rgb]{0.31,0.60,0.02}{#1}}
\newcommand{\VerbatimStringTok}[1]{\textcolor[rgb]{0.31,0.60,0.02}{#1}}
\newcommand{\SpecialStringTok}[1]{\textcolor[rgb]{0.31,0.60,0.02}{#1}}
\newcommand{\ImportTok}[1]{#1}
\newcommand{\CommentTok}[1]{\textcolor[rgb]{0.56,0.35,0.01}{\textit{#1}}}
\newcommand{\DocumentationTok}[1]{\textcolor[rgb]{0.56,0.35,0.01}{\textbf{\textit{#1}}}}
\newcommand{\AnnotationTok}[1]{\textcolor[rgb]{0.56,0.35,0.01}{\textbf{\textit{#1}}}}
\newcommand{\CommentVarTok}[1]{\textcolor[rgb]{0.56,0.35,0.01}{\textbf{\textit{#1}}}}
\newcommand{\OtherTok}[1]{\textcolor[rgb]{0.56,0.35,0.01}{#1}}
\newcommand{\FunctionTok}[1]{\textcolor[rgb]{0.00,0.00,0.00}{#1}}
\newcommand{\VariableTok}[1]{\textcolor[rgb]{0.00,0.00,0.00}{#1}}
\newcommand{\ControlFlowTok}[1]{\textcolor[rgb]{0.13,0.29,0.53}{\textbf{#1}}}
\newcommand{\OperatorTok}[1]{\textcolor[rgb]{0.81,0.36,0.00}{\textbf{#1}}}
\newcommand{\BuiltInTok}[1]{#1}
\newcommand{\ExtensionTok}[1]{#1}
\newcommand{\PreprocessorTok}[1]{\textcolor[rgb]{0.56,0.35,0.01}{\textit{#1}}}
\newcommand{\AttributeTok}[1]{\textcolor[rgb]{0.77,0.63,0.00}{#1}}
\newcommand{\RegionMarkerTok}[1]{#1}
\newcommand{\InformationTok}[1]{\textcolor[rgb]{0.56,0.35,0.01}{\textbf{\textit{#1}}}}
\newcommand{\WarningTok}[1]{\textcolor[rgb]{0.56,0.35,0.01}{\textbf{\textit{#1}}}}
\newcommand{\AlertTok}[1]{\textcolor[rgb]{0.94,0.16,0.16}{#1}}
\newcommand{\ErrorTok}[1]{\textcolor[rgb]{0.64,0.00,0.00}{\textbf{#1}}}
\newcommand{\NormalTok}[1]{#1}
\usepackage{graphicx,grffile}
\makeatletter
\def\maxwidth{\ifdim\Gin@nat@width>\linewidth\linewidth\else\Gin@nat@width\fi}
\def\maxheight{\ifdim\Gin@nat@height>\textheight\textheight\else\Gin@nat@height\fi}
\makeatother
% Scale images if necessary, so that they will not overflow the page
% margins by default, and it is still possible to overwrite the defaults
% using explicit options in \includegraphics[width, height, ...]{}
\setkeys{Gin}{width=\maxwidth,height=\maxheight,keepaspectratio}
\IfFileExists{parskip.sty}{%
\usepackage{parskip}
}{% else
\setlength{\parindent}{0pt}
\setlength{\parskip}{6pt plus 2pt minus 1pt}
}
\setlength{\emergencystretch}{3em}  % prevent overfull lines
\providecommand{\tightlist}{%
  \setlength{\itemsep}{0pt}\setlength{\parskip}{0pt}}
\setcounter{secnumdepth}{0}
% Redefines (sub)paragraphs to behave more like sections
\ifx\paragraph\undefined\else
\let\oldparagraph\paragraph
\renewcommand{\paragraph}[1]{\oldparagraph{#1}\mbox{}}
\fi
\ifx\subparagraph\undefined\else
\let\oldsubparagraph\subparagraph
\renewcommand{\subparagraph}[1]{\oldsubparagraph{#1}\mbox{}}
\fi

%%% Use protect on footnotes to avoid problems with footnotes in titles
\let\rmarkdownfootnote\footnote%
\def\footnote{\protect\rmarkdownfootnote}

%%% Change title format to be more compact
\usepackage{titling}

% Create subtitle command for use in maketitle
\newcommand{\subtitle}[1]{
  \posttitle{
    \begin{center}\large#1\end{center}
    }
}

\setlength{\droptitle}{-2em}
  \title{Tutorial: Meta-Analysis of Proportion in R}
  \pretitle{\vspace{\droptitle}\centering\huge}
  \posttitle{\par}
  \author{Pius S. Ekong}
  \preauthor{\centering\large\emph}
  \postauthor{\par}
  \predate{\centering\large\emph}
  \postdate{\par}
  \date{July 9, 2018}


\begin{document}
\maketitle

\subsection{Dataset}\label{dataset}

Dataset was compiled through a systematic review process

\subsection{Install packages required for the
analysis}\label{install-packages-required-for-the-analysis}

\begin{Shaded}
\begin{Highlighting}[]
\CommentTok{# install.packages("tidyverse")}
\CommentTok{# install.packages("meta")}
\CommentTok{# install.packages("metafor")}
\end{Highlighting}
\end{Shaded}

\subsection{Load library}\label{load-library}

\begin{Shaded}
\begin{Highlighting}[]
\KeywordTok{library}\NormalTok{(tidyverse)}
\end{Highlighting}
\end{Shaded}

\begin{verbatim}
## Warning: package 'tidyverse' was built under R version 3.4.4
\end{verbatim}

\begin{verbatim}
## -- Attaching packages ----------------------------------------- tidyverse 1.2.1 --
\end{verbatim}

\begin{verbatim}
## v ggplot2 2.2.1     v purrr   0.2.5
## v tibble  1.4.2     v dplyr   0.7.5
## v tidyr   0.8.1     v stringr 1.3.1
## v readr   1.1.1     v forcats 0.3.0
\end{verbatim}

\begin{verbatim}
## Warning: package 'tibble' was built under R version 3.4.4
\end{verbatim}

\begin{verbatim}
## Warning: package 'tidyr' was built under R version 3.4.4
\end{verbatim}

\begin{verbatim}
## Warning: package 'readr' was built under R version 3.4.4
\end{verbatim}

\begin{verbatim}
## Warning: package 'purrr' was built under R version 3.4.4
\end{verbatim}

\begin{verbatim}
## Warning: package 'dplyr' was built under R version 3.4.4
\end{verbatim}

\begin{verbatim}
## Warning: package 'stringr' was built under R version 3.4.4
\end{verbatim}

\begin{verbatim}
## Warning: package 'forcats' was built under R version 3.4.4
\end{verbatim}

\begin{verbatim}
## -- Conflicts -------------------------------------------- tidyverse_conflicts() --
## x dplyr::filter() masks stats::filter()
## x dplyr::lag()    masks stats::lag()
\end{verbatim}

\begin{Shaded}
\begin{Highlighting}[]
\KeywordTok{library}\NormalTok{(meta)}
\end{Highlighting}
\end{Shaded}

\begin{verbatim}
## Warning: package 'meta' was built under R version 3.4.4
\end{verbatim}

\begin{verbatim}
## Loading 'meta' package (version 4.9-2).
## Type 'help(meta)' for a brief overview.
\end{verbatim}

\begin{Shaded}
\begin{Highlighting}[]
\KeywordTok{library}\NormalTok{(metafor)}
\end{Highlighting}
\end{Shaded}

\begin{verbatim}
## Warning: package 'metafor' was built under R version 3.4.4
\end{verbatim}

\begin{verbatim}
## Loading required package: Matrix
\end{verbatim}

\begin{verbatim}
## 
## Attaching package: 'Matrix'
\end{verbatim}

\begin{verbatim}
## The following object is masked from 'package:tidyr':
## 
##     expand
\end{verbatim}

\begin{verbatim}
## Loading 'metafor' package (version 2.0-0). For an overview 
## and introduction to the package please type: help(metafor).
\end{verbatim}

\begin{verbatim}
## 
## Attaching package: 'metafor'
\end{verbatim}

\begin{verbatim}
## The following objects are masked from 'package:meta':
## 
##     baujat, forest, funnel, funnel.default, labbe, radial,
##     trimfill
\end{verbatim}

\subsection{Read the dataset into R}\label{read-the-dataset-into-r}

\begin{Shaded}
\begin{Highlighting}[]
\NormalTok{hidedata <-}\StringTok{ }\KeywordTok{read.csv}\NormalTok{(}\StringTok{"Hide_Data.csv"}\NormalTok{, }\DataTypeTok{sep=}\StringTok{","}\NormalTok{ , }\DataTypeTok{header=}\NormalTok{T, }\DataTypeTok{fill=}\NormalTok{T, }\DataTypeTok{na.string=}\StringTok{" "}\NormalTok{)}
\KeywordTok{head}\NormalTok{(hidedata,}\DecValTok{10}\NormalTok{)}
\end{Highlighting}
\end{Shaded}

\begin{verbatim}
##    ID         Study Study.setting    Study.design   Cattle Sample.Type
## 1   1  Dewell 2008a    Commercial Cross sectional Fed beef   Hide-Farm
## 2   2  Renter 2008a    Commercial Cross sectional Fed beef   Hide-Farm
## 3   3 Arthur 2007Ca    Commercial Cross sectional Fed beef   Hide-Farm
## 4   4 Arthur 2007Cb    Commercial Cross sectional Fed beef   Hide-Farm
## 5   5 Arthur 2007Cc    Commercial Cross sectional Fed beef   Hide-Farm
## 6   6  Arthur 2008c    Commercial Cross sectional Fed beef   Hide-Farm
## 7   7  Arthur 2008e    Commercial Cross sectional Fed beef   Hide-Farm
## 8   8  Arthur 2008b    Commercial Cross sectional Fed beef   Hide-Farm
## 9   9  Arthur 2008d    commercial Cross sectional Fed beef   Hide-Farm
## 10 10  Arthur 2008a    commercial Cross sectional Fed beef   Hide-Farm
##    Area.Swab.cm2. Number.Sample Number.Positive Prevalence Country
## 1           <1000           785              48 0.06110000     USA
## 2           <1000          1040              42 0.04038461  Canada
## 3           >1000            81              23 0.28400000     USA
## 4           >1000           149             109 0.73150000     USA
## 5           >1000            56              12 0.21430000     USA
## 6           >1000            94              73 0.77660000     USA
## 7           >1000            87              67 0.77010000     USA
## 8           >1000            97              62 0.63920000     USA
## 9           >1000            88              67 0.76140000     USA
## 10          >1000           128              54 0.42190000     USA
##    Serotype Season
## 1    O157   Summer
## 2   O157 H7 Summer
## 3   O157 H7 Summer
## 4   O157 H7 Summer
## 5   O157 H7 Summer
## 6   O157 H7 Summer
## 7   O157 H7 Summer
## 8   O157 H7 Summer
## 9   O157 H7 Summer
## 10  O157 H7 Summer
\end{verbatim}

\subsection{Data cleaning}\label{data-cleaning}

Look at data loaded as ``hidedata''

\begin{Shaded}
\begin{Highlighting}[]
\KeywordTok{glimpse}\NormalTok{(hidedata)}
\end{Highlighting}
\end{Shaded}

\begin{verbatim}
## Observations: 25
## Variables: 13
## $ ID              <int> 1, 2, 3, 4, 5, 6, 7, 8, 9, 10, 11, 12, 13, 14,...
## $ Study           <fct> Dewell 2008a, Renter 2008a, Arthur 2007Ca, Art...
## $ Study.setting   <fct> Commercial, Commercial, Commercial, Commercial...
## $ Study.design    <fct> Cross sectional, Cross sectional, Cross sectio...
## $ Cattle          <fct> Fed beef, Fed beef, Fed beef, Fed beef, Fed be...
## $ Sample.Type     <fct> Hide-Farm, Hide-Farm, Hide-Farm, Hide-Farm, Hi...
## $ Area.Swab.cm2.  <fct> <1000, <1000, >1000, >1000, >1000, >1000, >100...
## $ Number.Sample   <int> 785, 1040, 81, 149, 56, 94, 87, 97, 88, 128, 3...
## $ Number.Positive <int> 48, 42, 23, 109, 12, 73, 67, 62, 67, 54, 31, 4...
## $ Prevalence      <dbl> 0.06110000, 0.04038461, 0.28400000, 0.73150000...
## $ Country         <fct> USA, Canada, USA, USA, USA, USA, USA, USA, USA...
## $ Serotype        <fct> O157  , O157 H7, O157 H7, O157 H7, O157 H7, O1...
## $ Season          <fct> Summer, Summer, Summer, Summer, Summer, Summer...
\end{verbatim}

Replace the variable names in the dataset

\begin{Shaded}
\begin{Highlighting}[]
\KeywordTok{names}\NormalTok{(hidedata) <-}\StringTok{ }\KeywordTok{c}\NormalTok{(}\StringTok{"ID"}\NormalTok{, }\StringTok{"StudyName"}\NormalTok{, }\StringTok{"StudySetting"}\NormalTok{, }\StringTok{"StudyDesign"}\NormalTok{, }\StringTok{"CattleType"}\NormalTok{, }\StringTok{"SampleType"}\NormalTok{, }\StringTok{"AreaSwab"}\NormalTok{, }
                     \StringTok{"NumSample"}\NormalTok{, }\StringTok{"NumPositive"}\NormalTok{, }\StringTok{"PropPositive"}\NormalTok{, }\StringTok{"Country"}\NormalTok{, }\StringTok{"Serotype"}\NormalTok{, }\StringTok{"Season"}\NormalTok{)}
\KeywordTok{glimpse}\NormalTok{(hidedata)}
\end{Highlighting}
\end{Shaded}

\begin{verbatim}
## Observations: 25
## Variables: 13
## $ ID           <int> 1, 2, 3, 4, 5, 6, 7, 8, 9, 10, 11, 12, 13, 14, 15...
## $ StudyName    <fct> Dewell 2008a, Renter 2008a, Arthur 2007Ca, Arthur...
## $ StudySetting <fct> Commercial, Commercial, Commercial, Commercial, C...
## $ StudyDesign  <fct> Cross sectional, Cross sectional, Cross sectional...
## $ CattleType   <fct> Fed beef, Fed beef, Fed beef, Fed beef, Fed beef,...
## $ SampleType   <fct> Hide-Farm, Hide-Farm, Hide-Farm, Hide-Farm, Hide-...
## $ AreaSwab     <fct> <1000, <1000, >1000, >1000, >1000, >1000, >1000, ...
## $ NumSample    <int> 785, 1040, 81, 149, 56, 94, 87, 97, 88, 128, 38, ...
## $ NumPositive  <int> 48, 42, 23, 109, 12, 73, 67, 62, 67, 54, 31, 49, ...
## $ PropPositive <dbl> 0.06110000, 0.04038461, 0.28400000, 0.73150000, 0...
## $ Country      <fct> USA, Canada, USA, USA, USA, USA, USA, USA, USA, U...
## $ Serotype     <fct> O157  , O157 H7, O157 H7, O157 H7, O157 H7, O157 ...
## $ Season       <fct> Summer, Summer, Summer, Summer, Summer, Summer, S...
\end{verbatim}

\begin{Shaded}
\begin{Highlighting}[]
\KeywordTok{str}\NormalTok{(hidedata)}
\end{Highlighting}
\end{Shaded}

\begin{verbatim}
## 'data.frame':    25 obs. of  13 variables:
##  $ ID          : int  1 2 3 4 5 6 7 8 9 10 ...
##  $ StudyName   : Factor w/ 25 levels "Arthur 2007Ca",..: 20 25 1 2 3 6 8 5 7 4 ...
##  $ StudySetting: Factor w/ 3 levels "commercial","Commercial",..: 2 2 2 2 2 2 2 2 1 1 ...
##  $ StudyDesign : Factor w/ 2 levels "Cross sectional",..: 1 1 1 1 1 1 1 1 1 1 ...
##  $ CattleType  : Factor w/ 2 levels "Cull dairy","Fed beef": 2 2 2 2 2 2 2 2 2 2 ...
##  $ SampleType  : Factor w/ 2 levels "Hide-Farm","Hide-Plant": 1 1 1 1 1 1 1 1 1 1 ...
##  $ AreaSwab    : Factor w/ 2 levels "<1000",">1000": 1 1 2 2 2 2 2 2 2 2 ...
##  $ NumSample   : int  785 1040 81 149 56 94 87 97 88 128 ...
##  $ NumPositive : int  48 42 23 109 12 73 67 62 67 54 ...
##  $ PropPositive: num  0.0611 0.0404 0.284 0.7315 0.2143 ...
##  $ Country     : Factor w/ 2 levels "Canada","USA": 2 1 2 2 2 2 2 2 2 2 ...
##  $ Serotype    : Factor w/ 2 levels "O157  ","O157 H7": 1 2 2 2 2 2 2 2 2 2 ...
##  $ Season      : Factor w/ 2 levels "Summer","Winter": 1 1 1 1 1 1 1 1 1 1 ...
\end{verbatim}

Let's take a look at the summary of the data

\begin{Shaded}
\begin{Highlighting}[]
\KeywordTok{summary}\NormalTok{(hidedata)}
\end{Highlighting}
\end{Shaded}

\begin{verbatim}
##        ID             StudyName             StudySetting
##  Min.   : 1   Arthur 2007Ca: 1   commercial       :14   
##  1st Qu.: 7   Arthur 2007Cb: 1   Commercial       : 8   
##  Median :13   Arthur 2007Cc: 1   Research facility: 3   
##  Mean   :13   Arthur 2008a : 1                          
##  3rd Qu.:19   Arthur 2008b : 1                          
##  Max.   :25   Arthur 2008c : 1                          
##               (Other)      :19                          
##           StudyDesign      CattleType      SampleType  AreaSwab 
##  Cross sectional:24   Cull dairy:10   Hide-Farm :15   <1000: 2  
##  Field trial    : 1   Fed beef  :15   Hide-Plant:10   >1000:23  
##                                                                 
##                                                                 
##                                                                 
##                                                                 
##                                                                 
##    NumSample       NumPositive      PropPositive        Country  
##  Min.   :  38.0   Min.   : 10.00   Min.   : 0.04038   Canada: 1  
##  1st Qu.:  88.0   1st Qu.: 41.00   1st Qu.: 0.42190   USA   :24  
##  Median : 190.0   Median : 62.00   Median : 0.81580              
##  Mean   : 245.6   Mean   : 88.44   Mean   :19.03301              
##  3rd Qu.: 190.0   3rd Qu.: 99.00   3rd Qu.:33.15789              
##  Max.   :1040.0   Max.   :423.00   Max.   :90.00000              
##                                                                  
##     Serotype     Season  
##  O157   : 1   Summer:20  
##  O157 H7:24   Winter: 5  
##                          
##                          
##                          
##                          
## 
\end{verbatim}

The variable ``StudySetting'' is suppose to have 2 levels -
``Commercial'' \& ``Research Facility''

\begin{Shaded}
\begin{Highlighting}[]
\NormalTok{convert <-}\StringTok{ }\KeywordTok{which}\NormalTok{(hidedata}\OperatorTok{$}\NormalTok{StudySetting }\OperatorTok{==}\StringTok{"commercial"}\NormalTok{)}
\NormalTok{hidedata}\OperatorTok{$}\NormalTok{StudySetting[convert] <-}\StringTok{ "Commercial"}
\KeywordTok{summary}\NormalTok{(hidedata}\OperatorTok{$}\NormalTok{StudySetting)}
\end{Highlighting}
\end{Shaded}

\begin{verbatim}
##        commercial        Commercial Research facility 
##                 0                22                 3
\end{verbatim}

Let's get rid of the empty factor level `commercial'

\begin{Shaded}
\begin{Highlighting}[]
\NormalTok{hidedata}\OperatorTok{$}\NormalTok{StudySetting <-}\StringTok{ }\KeywordTok{droplevels}\NormalTok{(hidedata}\OperatorTok{$}\NormalTok{StudySetting)}
\KeywordTok{summary}\NormalTok{(hidedata}\OperatorTok{$}\NormalTok{StudySetting)}
\end{Highlighting}
\end{Shaded}

\begin{verbatim}
##        Commercial Research facility 
##                22                 3
\end{verbatim}

\section{Data Visualization}\label{data-visualization}

Categorical data

\begin{Shaded}
\begin{Highlighting}[]
\KeywordTok{par}\NormalTok{(}\DataTypeTok{mfrow =} \KeywordTok{c}\NormalTok{(}\DecValTok{2}\NormalTok{,}\DecValTok{4}\NormalTok{))}
\KeywordTok{barplot}\NormalTok{(Serotype <-}\StringTok{ }\KeywordTok{table}\NormalTok{(hidedata}\OperatorTok{$}\NormalTok{StudySetting))}
\KeywordTok{barplot}\NormalTok{(Serotype <-}\StringTok{ }\KeywordTok{table}\NormalTok{(hidedata}\OperatorTok{$}\NormalTok{StudyDesign))}
\KeywordTok{barplot}\NormalTok{(Serotype <-}\StringTok{ }\KeywordTok{table}\NormalTok{(hidedata}\OperatorTok{$}\NormalTok{CattleType))}
\KeywordTok{barplot}\NormalTok{(Serotype <-}\StringTok{ }\KeywordTok{table}\NormalTok{(hidedata}\OperatorTok{$}\NormalTok{SampleType))}
\KeywordTok{barplot}\NormalTok{(Serotype <-}\StringTok{ }\KeywordTok{table}\NormalTok{(hidedata}\OperatorTok{$}\NormalTok{AreaSwab))}
\KeywordTok{barplot}\NormalTok{(Serotype <-}\StringTok{ }\KeywordTok{table}\NormalTok{(hidedata}\OperatorTok{$}\NormalTok{Country))}
\KeywordTok{barplot}\NormalTok{(Serotype <-}\StringTok{ }\KeywordTok{table}\NormalTok{(hidedata}\OperatorTok{$}\NormalTok{Serotype))}
\KeywordTok{barplot}\NormalTok{(Serotype <-}\StringTok{ }\KeywordTok{table}\NormalTok{(hidedata}\OperatorTok{$}\NormalTok{Season))}
\end{Highlighting}
\end{Shaded}

\includegraphics{MetaHideSummer_files/figure-latex/catdata-1.pdf}

Numeric data

\begin{Shaded}
\begin{Highlighting}[]
\KeywordTok{par}\NormalTok{(}\DataTypeTok{mfrow =} \KeywordTok{c}\NormalTok{(}\DecValTok{1}\NormalTok{,}\DecValTok{3}\NormalTok{))}
\KeywordTok{hist}\NormalTok{(hidedata}\OperatorTok{$}\NormalTok{NumSample)}
\KeywordTok{hist}\NormalTok{(hidedata}\OperatorTok{$}\NormalTok{NumPositive)}
\KeywordTok{hist}\NormalTok{(hidedata}\OperatorTok{$}\NormalTok{PropPositive)}
\end{Highlighting}
\end{Shaded}

\includegraphics{MetaHideSummer_files/figure-latex/numdata-1.pdf}

A priori we believe proportion positive differ by CattleType, SampleType
and Season. Take a subset of data for fed beef sampled at the farm
`Hide-farm'

\begin{Shaded}
\begin{Highlighting}[]
\NormalTok{hidedata_FB_Farm <-}\StringTok{ }\KeywordTok{subset}\NormalTok{(hidedata, CattleType}\OperatorTok{==}\StringTok{"Fed beef"} \OperatorTok{&}\StringTok{ }\NormalTok{SampleType}\OperatorTok{==}\StringTok{"Hide-Farm"} \OperatorTok{&}\StringTok{ }\NormalTok{Season}\OperatorTok{==}\StringTok{"Summer"}\NormalTok{)}
\KeywordTok{dim}\NormalTok{(hidedata_FB_Farm)}
\end{Highlighting}
\end{Shaded}

\begin{verbatim}
## [1] 15 13
\end{verbatim}

\subsection{Meta-analysis}\label{meta-analysis}

\begin{Shaded}
\begin{Highlighting}[]
\CommentTok{# E. coli on hide for fed beef sampled at feedlot }
\NormalTok{hidemeta_FB_Farm_S <-}\StringTok{ }\KeywordTok{metaprop}\NormalTok{(NumPositive, NumSample, }\DataTypeTok{studlab=}\NormalTok{StudyName, }\DataTypeTok{sm=}\StringTok{"PFT"}\NormalTok{, }\DataTypeTok{data=}\NormalTok{hidedata_FB_Farm, }\DataTypeTok{method=}\StringTok{"Inverse"}\NormalTok{, }\DataTypeTok{method.tau=}\StringTok{"DL"}\NormalTok{)}
\KeywordTok{summary}\NormalTok{(hidemeta_FB_Farm_S)}
\end{Highlighting}
\end{Shaded}

\begin{verbatim}
## Number of studies combined: k = 15
## 
##                      proportion           95%-CI  z p-value
## Fixed effect model       0.2085 [0.1942; 0.2232] --      --
## Random effects model     0.5198 [0.3136; 0.7226] --      --
## 
## Quantifying heterogeneity:
## tau^2 = 0.1668; H = 10.98 [10.04; 12.00]; I^2 = 99.2% [99.0%; 99.3%]
## 
## Test of heterogeneity:
##        Q d.f. p-value
##  1686.83   14       0
## 
## Details on meta-analytical method:
## - Inverse variance method
## - DerSimonian-Laird estimator for tau^2
## - Freeman-Tukey double arcsine transformation
## - Clopper-Pearson confidence interval for individual studies
\end{verbatim}

\subsection{Forest plot}\label{forest-plot}

\begin{Shaded}
\begin{Highlighting}[]
\CommentTok{# Summary effect measure obtained from random effect model only}
\KeywordTok{forest.meta}\NormalTok{(hidemeta_FB_Farm_S, }\DataTypeTok{layout=}\StringTok{"RevMan5"}\NormalTok{, }\DataTypeTok{xlab=}\StringTok{"Proportion"}\NormalTok{, }\DataTypeTok{comb.r=}\NormalTok{T, }\DataTypeTok{comb.f=}\NormalTok{F, }\DataTypeTok{xlim =} \KeywordTok{c}\NormalTok{(}\DecValTok{0}\NormalTok{,}\DecValTok{1}\NormalTok{), }\DataTypeTok{fontsize=}\DecValTok{10}\NormalTok{, }\DataTypeTok{digits=}\DecValTok{3}\NormalTok{)}
\end{Highlighting}
\end{Shaded}

\includegraphics{MetaHideSummer_files/figure-latex/Forest plot-1.pdf}

\subsection{Small Study Effects}\label{small-study-effects}

Assessment for small study effects

\begin{Shaded}
\begin{Highlighting}[]
\KeywordTok{funnel.meta}\NormalTok{(hidemeta_FB_Farm_S)}
\end{Highlighting}
\end{Shaded}

\includegraphics{MetaHideSummer_files/figure-latex/Small study-1.pdf}

\begin{Shaded}
\begin{Highlighting}[]
\KeywordTok{metabias}\NormalTok{(hidemeta_FB_Farm_S)}
\end{Highlighting}
\end{Shaded}

\begin{verbatim}
## 
##  Linear regression test of funnel plot asymmetry
## 
## data:  hidemeta_FB_Farm_S
## t = 5.5885, df = 13, p-value = 8.794e-05
## alternative hypothesis: asymmetry in funnel plot
## sample estimates:
##        bias     se.bias       slope 
## 17.40489745  3.11441116 -0.04128047
\end{verbatim}

\subsection{Meta-regression:
Univariable}\label{meta-regression-univariable}

\begin{Shaded}
\begin{Highlighting}[]
\KeywordTok{metareg}\NormalTok{(hidemeta_FB_Farm_S, }\OperatorTok{~}\StringTok{ }\NormalTok{StudyDesign)}
\end{Highlighting}
\end{Shaded}

\begin{verbatim}
## 
## Mixed-Effects Model (k = 15; tau^2 estimator: DL)
## 
## tau^2 (estimated amount of residual heterogeneity):     0.1876 (SE = 0.1231)
## tau (square root of estimated tau^2 value):             0.4331
## I^2 (residual heterogeneity / unaccounted variability): 99.23%
## H^2 (unaccounted variability / sampling variability):   129.46
## R^2 (amount of heterogeneity accounted for):            0.00%
## 
## Test for Residual Heterogeneity: 
## QE(df = 13) = 1682.9844, p-val < .0001
## 
## Test of Moderators (coefficient(s) 2): 
## QM(df = 1) = 0.8491, p-val = 0.3568
## 
## Model Results:
## 
##                         estimate      se     zval    pval    ci.lb   ci.ub
## intrcpt                   0.8331  0.1166   7.1421  <.0001   0.6044  1.0617
## StudyDesignField trial   -0.4143  0.4496  -0.9215  0.3568  -1.2956  0.4669
##                            
## intrcpt                 ***
## StudyDesignField trial     
## 
## ---
## Signif. codes:  0 '***' 0.001 '**' 0.01 '*' 0.05 '.' 0.1 ' ' 1
\end{verbatim}

\begin{Shaded}
\begin{Highlighting}[]
\KeywordTok{metareg}\NormalTok{(hidemeta_FB_Farm_S, }\OperatorTok{~}\StringTok{ }\NormalTok{StudySetting)}
\end{Highlighting}
\end{Shaded}

\begin{verbatim}
## 
## Mixed-Effects Model (k = 15; tau^2 estimator: DL)
## 
## tau^2 (estimated amount of residual heterogeneity):     0.1366 (SE = 0.0862)
## tau (square root of estimated tau^2 value):             0.3696
## I^2 (residual heterogeneity / unaccounted variability): 99.02%
## H^2 (unaccounted variability / sampling variability):   101.60
## R^2 (amount of heterogeneity accounted for):            18.10%
## 
## Test for Residual Heterogeneity: 
## QE(df = 13) = 1320.8590, p-val < .0001
## 
## Test of Moderators (coefficient(s) 2): 
## QM(df = 1) = 1.8546, p-val = 0.1732
## 
## Model Results:
## 
##                                estimate      se    zval    pval    ci.lb
## intrcpt                          0.7393  0.1077  6.8665  <.0001   0.5283
## StudySettingResearch facility    0.3292  0.2418  1.3618  0.1732  -0.1446
##                                 ci.ub     
## intrcpt                        0.9503  ***
## StudySettingResearch facility  0.8031     
## 
## ---
## Signif. codes:  0 '***' 0.001 '**' 0.01 '*' 0.05 '.' 0.1 ' ' 1
\end{verbatim}

\begin{Shaded}
\begin{Highlighting}[]
\KeywordTok{metareg}\NormalTok{(hidemeta_FB_Farm_S, }\OperatorTok{~}\StringTok{ }\NormalTok{AreaSwab)}
\end{Highlighting}
\end{Shaded}

\begin{verbatim}
## 
## Mixed-Effects Model (k = 15; tau^2 estimator: DL)
## 
## tau^2 (estimated amount of residual heterogeneity):     0.0645 (SE = 0.0435)
## tau (square root of estimated tau^2 value):             0.2539
## I^2 (residual heterogeneity / unaccounted variability): 97.60%
## H^2 (unaccounted variability / sampling variability):   41.64
## R^2 (amount of heterogeneity accounted for):            61.35%
## 
## Test for Residual Heterogeneity: 
## QE(df = 13) = 541.3725, p-val < .0001
## 
## Test of Moderators (coefficient(s) 2): 
## QM(df = 1) = 11.8529, p-val = 0.0006
## 
## Model Results:
## 
##                estimate      se    zval    pval    ci.lb   ci.ub     
## intrcpt          0.2272  0.1799  1.2628  0.2067  -0.1255  0.5799     
## AreaSwab>1000    0.6674  0.1939  3.4428  0.0006   0.2875  1.0474  ***
## 
## ---
## Signif. codes:  0 '***' 0.001 '**' 0.01 '*' 0.05 '.' 0.1 ' ' 1
\end{verbatim}

\begin{Shaded}
\begin{Highlighting}[]
\KeywordTok{metareg}\NormalTok{(hidemeta_FB_Farm_S, }\OperatorTok{~}\StringTok{ }\NormalTok{Country)}
\end{Highlighting}
\end{Shaded}

\begin{verbatim}
## 
## Mixed-Effects Model (k = 15; tau^2 estimator: DL)
## 
## tau^2 (estimated amount of residual heterogeneity):     0.1779 (SE = 0.0979)
## tau (square root of estimated tau^2 value):             0.4218
## I^2 (residual heterogeneity / unaccounted variability): 98.93%
## H^2 (unaccounted variability / sampling variability):   93.43
## R^2 (amount of heterogeneity accounted for):            0.00%
## 
## Test for Residual Heterogeneity: 
## QE(df = 13) = 1214.5327, p-val < .0001
## 
## Test of Moderators (coefficient(s) 2): 
## QM(df = 1) = 2.1794, p-val = 0.1399
## 
## Model Results:
## 
##             estimate      se    zval    pval    ci.lb   ci.ub   
## intrcpt       0.2035  0.4221  0.4821  0.6298  -0.6238  1.0307   
## CountryUSA    0.6453  0.4371  1.4763  0.1399  -0.2114  1.5020   
## 
## ---
## Signif. codes:  0 '***' 0.001 '**' 0.01 '*' 0.05 '.' 0.1 ' ' 1
\end{verbatim}

\begin{Shaded}
\begin{Highlighting}[]
\KeywordTok{metareg}\NormalTok{(hidemeta_FB_Farm_S, }\OperatorTok{~}\StringTok{ }\NormalTok{Serotype)}
\end{Highlighting}
\end{Shaded}

\begin{verbatim}
## 
## Mixed-Effects Model (k = 15; tau^2 estimator: DL)
## 
## tau^2 (estimated amount of residual heterogeneity):     0.2041 (SE = 0.1193)
## tau (square root of estimated tau^2 value):             0.4518
## I^2 (residual heterogeneity / unaccounted variability): 99.12%
## H^2 (unaccounted variability / sampling variability):   113.07
## R^2 (amount of heterogeneity accounted for):            0.00%
## 
## Test for Residual Heterogeneity: 
## QE(df = 13) = 1469.8669, p-val < .0001
## 
## Test of Moderators (coefficient(s) 2): 
## QM(df = 1) = 1.6116, p-val = 0.2043
## 
## Model Results:
## 
##                  estimate      se    zval    pval    ci.lb   ci.ub   
## intrcpt            0.2510  0.4521  0.5552  0.5787  -0.6351  1.1372   
## SerotypeO157 H7    0.5944  0.4682  1.2695  0.2043  -0.3233  1.5120   
## 
## ---
## Signif. codes:  0 '***' 0.001 '**' 0.01 '*' 0.05 '.' 0.1 ' ' 1
\end{verbatim}

\subsection{Meta-regression:
Multivariable}\label{meta-regression-multivariable}

\begin{Shaded}
\begin{Highlighting}[]
\KeywordTok{metareg}\NormalTok{(hidemeta_FB_Farm_S, }\OperatorTok{~}\StringTok{ }\NormalTok{StudySetting }\OperatorTok{+}\StringTok{ }\NormalTok{AreaSwab }\OperatorTok{+}\StringTok{ }\NormalTok{Country)}
\end{Highlighting}
\end{Shaded}

\begin{verbatim}
## 
## Mixed-Effects Model (k = 15; tau^2 estimator: DL)
## 
## tau^2 (estimated amount of residual heterogeneity):     0.1052 (SE = 0.0508)
## tau (square root of estimated tau^2 value):             0.3244
## I^2 (residual heterogeneity / unaccounted variability): 97.61%
## H^2 (unaccounted variability / sampling variability):   41.84
## R^2 (amount of heterogeneity accounted for):            36.91%
## 
## Test for Residual Heterogeneity: 
## QE(df = 11) = 460.2710, p-val < .0001
## 
## Test of Moderators (coefficient(s) 2:4): 
## QM(df = 3) = 8.3872, p-val = 0.0387
## 
## Model Results:
## 
##                                estimate      se    zval    pval    ci.lb
## intrcpt                          0.2035  0.3248  0.6264  0.5310  -0.4331
## StudySettingResearch facility    0.2253  0.2173  1.0370  0.2997  -0.2005
## AreaSwab>1000                    0.5925  0.3411  1.7368  0.0824  -0.0761
## CountryUSA                       0.0476  0.4594  0.1036  0.9175  -0.8528
##                                 ci.ub   
## intrcpt                        0.8400   
## StudySettingResearch facility  0.6512   
## AreaSwab>1000                  1.2611  .
## CountryUSA                     0.9480   
## 
## ---
## Signif. codes:  0 '***' 0.001 '**' 0.01 '*' 0.05 '.' 0.1 ' ' 1
\end{verbatim}

\section{Reference:}\label{reference}

Details of the systematic review and meta-analysis process and results
was published here. Ekong P.S., Sanderson M.W., Cernicchiaro N., 2015.
Prevalence and concentration of Escherichia coli O157 in different
seasons and cattle types processed in North America: A systematic review
and meta-analysis of published research. Prev. Vet. Med. 121, 74â€``85


\end{document}
